\documentclass[a4paper,10pt]{article}
\usepackage[T1]{fontenc}
\usepackage[utf8]{inputenc}
\usepackage{lmodern}
\usepackage[sort, numbers]{natbib}
\usepackage{graphicx}
\usepackage{amsmath}


\title{Temperature and energy measurements in ultra-low temperature Helium-3}
\author{PF}

\begin{document}

\maketitle
%\tableofcontents

%\begin{abstract}
%\end{abstract}

%\section{}

Specific heat capacity $C_V$ as a function of the temperature $T$ (in units of [J/K/m$^3$]) has been calculated from the entropy of a system of independent fermions, quasiparticles~\cite{vollhardt}, as

\begin{equation}
C_V(T)=2(2\pi)^{1/2}k_BN(0)\Delta\left(\frac{\Delta}{k_BT}\right)\exp\left(-\frac{\Delta}{k_BT}\right)
\end{equation}

where $k_B$ is the Boltzmann constant, $N(0)$ is the density of quasiparticle states in the normal phase at Fermi energy for one spin component, $\Delta$ ($\approx 1.76 k_B T_c$) is the average gap energy at low temperature near the critical temperature for superfluidity $T_c$ ($\simeq$930\,$\mu$K).

$C_V(T)$ has a strong dependency with temperature (fig.~\ref{fig:CvT})

\begin{figure}[htb]
  \begin{center}
    \includegraphics[width=0.69\textwidth]{Cv_vs_T-zoom}
  \end{center}
  \caption{$C_V$ vs T, below $T_c$. He3 has an extremely low heat capacity at low temperatures.}
  \label{fig:CvT}
\end{figure}

To obtain the heat variation $\Delta Q$ between two temperatures $T_1$ and $T_2$ for a a given volume $V$ of helium-3 is required the integration

\begin{align}
\Delta Q (T_1,T_2) = \Delta C_V V & = V \int_{T_1}^{T_2} C_V(T)dT \\
                                  & = V \left( \int_{0}^{T_2} C_V(T)dT - \int_{0}^{T_1} C_V(T)dT \right)\\
                                  & = V \left( I(T_2) - I(T_1) \right)  
\end{align}

where

\begin{align}
I(T) = \int_{0}^{T} C_V(T)dT & = \sqrt{2} \pi N(0) \Delta^2 \left( 1 - \frac{2}{\sqrt{\pi}}\int_{0}^{\sqrt \frac{\Delta}{k_BT}} e^{-t^2} dt \right) \\ 
                             & =  \sqrt{2}\pi N(0)\Delta^2 \mathrm{erfc} \sqrt{\frac{\Delta}{k_BT} } 
\end{align}

therefore

\begin{equation}
\Delta Q (T_1,T_2) = \sqrt{2}\pi N(0)\Delta^2 \left( \mathrm{erfc}\sqrt{\frac{\Delta}{k_BT_2}} - \mathrm{erfc}\sqrt{\frac{\Delta}{k_BT_1}} \right)V
\end{equation}

We consider the 1\,cm$^3$ volume of helium-3 at a certain base temperature $T_0$ so we can calculate the temperature variation $\Delta T$ for an energy deposition $\Delta Q(T_0,T_0+\Delta T)$;
since both $N(0)$ and $\Delta$ are function of the pressure we calculate the energy deposition for 4 different pressures in the range (0--30)\,bar (fig.~\ref{fig:TDE})

\begin{figure}[!ht]
  \begin{center}
  \begin{tabular}{cc}
    \includegraphics[width=0.49\textwidth]{T_vs_DE-0bar} &
    \includegraphics[width=0.49\textwidth]{T_vs_DE-5bar} \\
    \includegraphics[width=0.49\textwidth]{T_vs_DE-10bar} &
    \includegraphics[width=0.49\textwidth]{T_vs_DE-30bar}
  \end{tabular}
  \end{center}
  \caption{$\Delta T$ vs $\Delta Q$ for several $T_0$ in the range 40--200\,$\mu$K for pressures of 0\,bar (top-left), 5\,bar (top-right), 10\,bar (bottom-left), 30\,bar (bottom-right).}
  \label{fig:TDE}
\end{figure}

Higher pressures for higher system temperatures have equivalent effects in terms of temperature variations.

For a vibrating wire resonator (VWR), either from an amplitude sweep or a resonance tracking is possible to calculate the \textit{resonance width} $\Delta f$ which can be used to infer the temperature~\cite{lawson}.
Since the resonance width is
\begin{equation}
\Delta f = \frac{2F}{\pi \rho d}
\end{equation}

where $d$ and $\rho$ are respectively the diameter and the mass density of the wire, and $F$ is the damping(?) force (in units of length, velocity and diameter), defined as

\begin{equation}
F = \frac{\pi}{4}p^2_F v_F N(0) \exp{\left( -\frac{\Delta}{k_B T} \right)}
\end{equation}

The resonance width expressed in terms of temperature is

\begin{equation}
\Delta f = \frac{p^2_F v_F N(0)}{2\pi\rho d}\exp{\left( -\frac{\Delta}{k_B T} \right)}
\end{equation}

e.g. in fig.~\ref{fig:WvsT} for a Niobium-Titanium wire of 150\,nm diameter in a volume at 1\,bar; therefore the temperature can be derived from the resonance width

\begin{figure}[!ht]
  \begin{center}
    \includegraphics[width=0.69\textwidth]{W_vs_T}
    \caption{Resonance frequency width dependence on temperature for a nano-wire.}
    \label{fig:WvsT}
  \end{center}
\end{figure}

\begin{equation}
T = - \frac{\Delta}{k_B \ln \left( \Delta f \frac{2 \pi \rho d}{p_F^2 v_F N(0)} \right)}
\end{equation}

The increase of resonance width $\Delta (\Delta f)$ in response of an energy deposition $\Delta Q$ is linearly proportional, as shown in figure~\ref{fig:DeltaDeltaWvsDE}.
The constant of proportionality is inversely proportional to the base temperature (and pressure), so the other way around, in terms of energy deposition
\begin{equation}
  \Delta Q = \alpha(T_0,P) \Delta (\Delta f), \alpha \propto T_0
    \label{eq:alpha}
\end{equation}

\begin{figure}[!ht]
  \begin{center}
  \begin{tabular}{cc}
    \includegraphics[width=0.49\textwidth]{DeltaDeltaW_vs_DE-0bar}  &
    \includegraphics[width=0.49\textwidth]{DeltaDeltaW_vs_DE-5bar}  \\
    \includegraphics[width=0.49\textwidth]{DeltaDeltaW_vs_DE-10bar} &
    \includegraphics[width=0.49\textwidth]{DeltaDeltaW_vs_DE-30bar}
  \end{tabular}
  \end{center}
  \caption{$\Delta(\Delta f)$ vs $\Delta Q$ for several $T_0$ in the range 50--300\,$\mu$K for pressures of 0\,bar (top-left), 5\,bar (top-right), 10\,bar (bottom-left), 30\,bar (bottom-right).}
  \label{fig:DeltaDeltaWvsDE}
\end{figure}

The increase of resonance width $\Delta (\Delta f)$ can be measured from fitting each bolometric recorded event, e.g. in figure~\ref{fig:winkelmann} from~\cite{winkelmann},
\begin{figure}[!ht]
  \begin{center}
    \begin{tabular}{cc}
    \includegraphics[width=0.42\textwidth]{winkelmann} &
    \includegraphics[width=0.49\textwidth]{winkelmann_fit.pdf}
    \end{tabular}
  \end{center}
  \caption{Example of an energy deposition event for a 4.5\,$\mu$m wire in a volume of 0.14\,cm$^3$ of He-3. H in the paper's notation is proportional to $\Delta (\Delta f)$ with respect to a base width~\cite{winkelmann} (\textit{left}); fit function as in eq.~\ref{fit} (\textit{right}) with $\tau_b$=5\,s, $\tau_w$=0.77\,s and $\Delta f_\mathrm{base}$=414\,mHz.}
  \label{fig:winkelmann}
\end{figure}
using the function

\begin{equation}
  \Delta f(t)= \Delta f_\mathrm{base} + \Delta (\Delta f) {\left( \frac{\tau_b}{\tau_w} \right)}^{\tau_w/(\tau_b-\tau_w)} \frac{\tau_b}{\tau_b - \tau_w} \left( e^{-t/\tau_b} - e^{-t/\tau_w} \right)
\label{fit}
\end{equation}

in order to extract the maximum variation $\Delta (\Delta f)$, where $\Delta f_\mathrm{base}$ is the base width at the base temperature of the helium, $\tau_w$ is the response time of the oscillating wire
\begin{equation}
  \tau_w \simeq \frac{1}{\pi \Delta f} = \mathrm{const}
\end{equation}
and $\tau_b$ is the decay constant, proportional to the Kapitza resistance (the thermal boundary resistance limiting the heat conduction between the solid metal and the liquid helium)
\begin{equation}
  \tau_b = R_K(T) C_V
\end{equation}

Considering that there are three main variables, base temperature, pressure of the helium, wire diameter (considering fixed the wire material and the helium volume), all with an interplay between each other, we can at least extract few conclusions:
\begin{itemize}
  \item For small wires there is an higher amplitude of the width response $\Delta (\Delta f)$
  \item For low temperatures there is an higher amplitude of the width response $\Delta (\Delta f)$
  \item For high pressures there is a lower width response for a certain energy deposition
  \item The response time of the oscillating wire is faster for high temperatures 
  \item The decay time of the oscillating wire is faster for high temperatures (low Kapitza resistance), so low temperatures might cause some pile-up of events

\end{itemize}

Ultimately the measurement done in the lab is done in terms of voltage readout of the oscillating wire, in two main steps:
\begin{enumerate}

  \item \textit{Sweep measurement} to extract the resonance frequency of the wire for a certain base temperature and the base width $\Delta f_\mathrm{base}$. The quantities are extracted from the Lorentzian fits of the in-phase and out-of-phase signal (Voltage vs Frequency). This should provide the error $\sigma_{\Delta f_\mathrm{base}}$.
  \item Data acquisition with the setup sitting at the resonance frequency (\textit{resonance tracking}) for a fixed voltage drive value $V_D$ (amplitude of the injected voltage). The measurement is voltage height $V_H$. From the Lorentzian function we know that
  \begin{equation}
    \frac{V_H\Delta f}{V_D} = \mathrm{const} = K
  \end{equation}
in particular for the base values
  \begin{equation}
    \frac{V_{H_\mathrm{base}}\Delta f_\mathrm{base}}{V_D} = \mathrm{const} = K
  \end{equation}
so we are going to measure $\Delta f(t)$, as in figure~\ref{fig:winkelmann}, as
  \begin{equation}
    \Delta f(t) = \frac{V_D}{V_H(t)}K
  \end{equation}
  
\end{enumerate}

% ============================================

\section{Errors}

In order to answer to the following questions, we need to find which is the measurement uncertainty coming from the laboratory setup:
\begin{itemize}

  \item What is the minimum voltage variation we can measure, hence the threshold of the energy deposition? 
  \item Which is the resolution in the resonance width variation (aka temperature variation), hence the resolution of the energy deposition?

\end{itemize}

We assume that the main uncertainty comes from the voltage measurement $\sigma_{V_H}$ which propagates into the determination of $\Delta f_\mathrm{base}$
and into the quality of the fit to extract $\Delta (\Delta f)$

The error on the energy deposition should be

\begin{equation}
  \sigma_{\Delta Q}  \simeq \left| \frac{\partial \Delta Q}{\partial(\Delta(\Delta f))} \right| \sigma_{\Delta (\Delta f)} \overset{\mathrm{eq. \ref{eq:alpha}}}{=} \alpha (T_0,P) \sigma_{\Delta (\Delta f)}
\end{equation}

The analytical error propagation might prove to be ineffective; is worth assigning an error coming from the voltage measurement noise from the lock-in amplifier and do a systematic error study starting from a toy distribution generated $\Delta f(t)$ (where the noise is present both for the baseline and for the signal distribution); find an error on the fit done to extract $\Delta (\Delta f)$; propagate this error to $\Delta Q$.

Since the data acquisition will fit $\Delta f$ we need to add the uncertainty $\sigma_{\Delta f}$ on the toy generated function from eq.~\ref{fit}, where
\begin{equation}
\sigma_{\Delta f}(t) = \left| \frac{\partial \Delta f(t)}{\partial V_H(t)} \right| \sigma_{V_H} = \frac{(\Delta f(t))^2}{V_{H_{\mathrm{base}}} \Delta f_\mathrm{base}} \sigma_{V_H}
\end{equation}
producing a distribution like in~figure~\ref{fig:toy}. To summarize the independent parameters present in the simulation are

\begin{itemize}
  \item volume and pressure of the helium-3 cell (V, P)
  \item wire density ($\rho$)
  \item diameter of the oscillating wire (d)
  \item base temperature of the helium ($T_0$)
  \item decay constant and response time ($\tau_b$, $\tau_w$)
  \item base voltage height ($V_{H_\mathrm{base}}$)
  \item error on the voltage measurement ($\sigma_{V_H}$)
\end{itemize}

\begin{figure}[!ht]
  \begin{center}
    \includegraphics[width=0.69\textwidth]{deltaf_toy-example}
    \caption{Example of a pseudo-experiment distribution of $\Delta f(t)$.}
    \label{fig:toy}
  \end{center}
\end{figure}

The randomization and fit of the pseudo-experiments is repeated N times, ultimately extracting the distribution of the deposited energy from the fitted $\Delta(\Delta f)$ using eq.~\ref{eq:alpha}, as in figure~\ref{fig:energy_distribution}.

\begin{figure}[!ht]
  \begin{center}
    \includegraphics[width=0.69\textwidth]{energy_distribution}
    \caption{Energy distribution for 10000 toys.}
    \label{fig:energy_distribution}
  \end{center}
\end{figure}

The error for the energy measurement can be related to $\sigma/\mu$, so for a range of energies [0--100]\,KeV is possible to obtain distributions of the expected error for a certain configuration (figure~\ref{fig:error})

\begin{figure}[!ht]
  \begin{center}
    \begin{tabular}{cc}
    \includegraphics[width=0.49\textwidth]{error_low} &
    \includegraphics[width=0.49\textwidth]{error}
    \end{tabular}
    \caption{Error vs energy for a certain configuration. The relative error increases with energy because is proportional to the square of the width increase.}
    \label{fig:error}
  \end{center}
\end{figure}

% ============================================

\pagebreak
\newpage

\begin{thebibliography}{9}

\bibitem{vollhardt} Vollhardt, D., Wolfle, P., The Superfluid Phases of Helium 3 (1990)
\bibitem{lawson} Lawson, C.R., A Novel Measurement Device for use in Multiphase Helium-3 and 4 at Ultra-Low Temperatures, MPHys thesis (2014)
\bibitem{winkelmann} Winkelmann et al, Bolometric calibration of a superfluid 3He detector for Dark Matter search: Direct measurement of the scintillated energy fraction for
neutron, electron and muon events (2007)

\end{thebibliography}

\end{document}
